\usepackage[compat2,pdftex]{geometry}
\usepackage[pdftex]{geometry}
\geometry{headsep=-2.5ex,hscale=0.9,textwidth=25cm}
\usepackage[pdftex]{xcolor}
\usepackage{pause}
\usepackage{background}
%\usepackage[color=white]{background}
\usepackage{listings}
\usepackage{graphicx}
\usepackage{epsfig}
\usepackage{epstopdf}
\usepackage{bm}
\usepackage{amssymb}
\usepackage{amsfonts}
\usepackage{marvosym}
\usepackage{cite}
\usepackage{pifont}
\usepackage[english]{babel}
\usepackage{amsmath}
\usepackage{amsfonts}
\usepackage{rotating}
\usepackage{listings}
\usepackage{pp4slide}
%\usepackage{hyperref}
\usepackage{pp4link}
\usepackage{fancyvrb}
\usepackage{multido}
\usepackage{calc}

\renewcommand{\ttdefault}{pcr}

\newcommand{\squeeze}{\setlength{\itemsep}{-2pt}}
%\renewcommand{\familydefault}{cmr}
%\renewcommand{\labelitemii}{\textcolor{green}{$\star$}}
\newcommand{\tick}{{\large \ding{51}}}
\newcommand{\cross}{{\large \ding{55}}}
%\DeclareMathAlphabet{\mat}{OT1}{cmss}{bx}{n}
\DeclareMathAlphabet{\mat}{U}{eur}{b}{n}
\newcommand{\figs}{/home/apb/figures}
\newcommand{\stimes}{{\scriptstyle \times}}
\newcommand{\tr}{\textsf{T}}
\newcommand{\e}[1]{{\rm e}^{#1}}
\newcommand{\E}[1]{{\rm e}^{{\displaystyle #1}}}
\newcommand{\av}[2][]{\mathbb{E}_{#1\!}\left[ #2 \right]}
\newcommand{\Var}[1]{\mathbb{V}\mathrm{ar}\left( #1 \right)}
\newcommand{\pred}[1]{\left\llbracket { \small #1} \right\rrbracket}
\newcommand{\logg}[1]{\log\!\left( #1 \right)}
\newcommand{\prob}[1]{\mathbb{P}\left( #1 \right)}
\newcommand{\Prob}[1]{\mathbb{P}\left( #1 \right)}
\newcommand{\Step}{\mathop{\mat{\Theta}}}
\newcommand\diag{\mathop{\mathrm{diag}}}
\newcommand{\grad}{\bm{\nabla}}
\newcommand{\gradx}{\bm{\nabla}_{\!\!\scriptscriptstyle \bm{x}}}
\newcommand{\gradw}{\bm{\nabla}_{\!\!\scriptscriptstyle \bm{w}}}
\newcommand{\dd}{\mathrm{d}}
\newcommand{\data}{\mathcal{D}}
\newcommand{\class}{\mathcal{C}}
\newcommand{\hypo}{\mathcal{H}}
\newcommand{\pd}[3][\,]{\frac{\partial^{#1}{#2}}{\partial\,{#3}}}
\newcommand{\normal}[1]{\mathcal{N}\!\left( #1 \right)}
\newcommand{\Dist}[2][Binom]{\mathrm{#1}\left( \strut {#2} \right)}
\newcommand{\brackets}[1]{\!\left( #1 \right)}
\newcommand{\KL}[2]{\mathop{\mathrm{KL}}\left(#1 \big\| #2\right)}
\lstnewenvironment{matlab}[1][frame=TB]{\lstset{#1}}{}       
%\renewcommand{\emph}[1]{\textcolor{red}{\textbf{#1}}}
\newcommand{\myskip}[1]{}
\newcommand{\jl}{\lstinline[basicstyle=\ttfamily\color{DarkGreen},%
 keywordstyle=\bfseries\ttfamily\color{DarkGreen}]}
\definecolor{EmphColor}{rgb}{0.0,0.0,1.0}
\renewcommand{\emph}[1]{{\color{EmphColor}\textbf{#1}}}
\newenvironment{psmallmatrix}
  {\left(\begin{smallmatrix}}
  {\end{smallmatrix}\right)}

\definecolor{DarkGreen}{rgb}{0,0.5,0}
\definecolor{DarkRed}{rgb}{0,0,0.5}
\definecolor{HighLight}{rgb}{1,0,0}     % yellow
\definecolor{TextColor}{rgb}{0.0,0.1,0.1}     % white
\definecolor{TwoColor}{rgb}{0.0,0.5,0.1}      % used for switching colors

\definecolor{HeadLineColor}{rgb}{1.0,0.0,0.0} % cyan
\definecolor{SpecialColor}{rgb}{0.53,0.81,1.0}
\renewcommand{\normalcolor}{\color{HeadLineColor}}

\pausecolors{TwoColor}{TextColor}{HighLight}
\pausecolors{EmphColor}{TextColor}{EmphColor}
\pausecolors{DarkGreen}{DarkGreen}{DarkRed}
\pausecolors{SpecialColor}{white}{HighLight}

\newenvironment{PauseHighLight}{\color{TwoColor}\pausehighlight}{}
\newcommand{\multipdf}[2][width=\linewidth]{\multiinclude[pause=\pauselevel{:+0}\pause,format=pdf,graphics={#1}]{#2}}

\lstdefinelanguage{pseudo}
 {keywords={all,and,begin,break,continue,do,else,elseif,%
     end,endif,endfor,endwhile,%
     exit,false,for,forall,goto,if,%
     or,return,then,true,to,repeat,until,while},%
  comment=[s]{/*}{*/},%
  morecomment=[l]//,%
  morestring=[d]"%
 }[keywords,comments,strings]


\lstloadlanguages{Java,pseudo}
\lstset{
  basicstyle=\footnotesize\ttfamily\color{DarkGreen},
  keywordstyle=\bfseries\footnotesize\ttfamily\color{DarkGreen},
  commentstyle=\footnotesize\it\rmfamily\color{blue},
  language=Java,
  }


\lstnewenvironment{pseudo}[1][]{\lstset{language=pseudo,mathescape,literate={:=}{{$\gets$}}2 {!}{{$\neg$}}1 {!=}{{\!\!$\neq$\,\,}}2 {<=}{{\!\!$\leq$\,\,}}2 {>=}{{\!\!$\geq$\,\,}}2,escapechar=\#,#1}}{}
\lstnewenvironment{java}[1][]{\lstset{escapechar=\$,#1}}{}

\MyLogo{}
%\Restriction{}
%\rightfooter{\hyperlink{overview}{$\square$}}

\newcommand{\keywords}[1]{\vfill{\noindent\it #1}}

\newcounter{outlineitem}
\setcounter{outlineitem}{0}
\newcommand{\outlineitem}[2]{\item%
  \def\targ{\value{emumi}}%
  \ifnum\value{enumi}=\value{outlineitem}%
  \color{HighLight}\textbf{#1}\toptarget{#2}\color{TextColor}%
   \else\toplink{#2}{#1}\fi
}
\color{TwoColor}


%-- Instead of writing \section we want the file structured by \section.
\newcommand{\section}[2][0]{\foilhead[#1cm]{#2}}

%-- Write each page into a \begin{slide} -- \end{slide} environment.
\newenvironment{slide}{}{\pauselevel{=1}}

\MyLogo{\pauselevel{=1}\Acrobatmenu{GoBack}{%
  \color{TextColor}{Adam Pr\"ugel-Bennett}} \hspace{4cm}
  % Clicking on the name steps one slide back.
  \toplink{firstoutline}{\color{red}COMP6208 Advanced Machine Learning}%
  %Clicking on the date jumps to the last page, which should be the
  %table of contents.
  \rightfooter{\Acrobatmenu{LastPage}{\qquad%
  \textsf{\color{TextColor}\tiny\thepage}}}% page number
  % Clicking on the page number steps one slide back.
}
\raggedright
\newcommand{\Outline}{}

\newcommand{\pb}{\pausebuild \color{TwoColor}}
\newcommand{\pauseb}{\pauselevel{build, highlight}\pause}
\newcommand{\hl}[1]{\pauselevel{=1, highlight =1 :1, highlight =#1 :#1}\pause}
\newcommand{\hll}{\pauselevel{=1, highlight =1 :1}\pause}
\newcommand{\pauseh}{\pauselevel{highlight}\pause}
\newcommand{\nhl}[1]{\textcolor{TextColor}{#1}}
\newcommand{\mypl}[1]{\pauselevel{=#1 :#1}\pause}
\input supp-pdf.tex
\DeclareGraphicsRule{*}{mps}{*}{}
\usepackage{mpmulti}
%\leftheader{}
\rightheader{} % no more page numbers on top right


\hypersetup{pdftitle={ECS Lectures},pdfsubject={Machine Learning}
pdfauthor={Adam Prugel-Bennett},pdfkeywords={machine learning},
pdfpagemode={FullScreen},
bookmarksopen=false,
hypertexnames=false,pdfstartview={FitV},colorlinks,linkcolor={blue},
citecolor={green},urlcolor={blue}}



\newcommand\bluepage{
\clearpage
\renewcommand\normalcolor{\color{yellow}}
\pagecolor{blue}
\renewcommand\Black{\color{white}}
\color{white}
\hypersetup{colorlinks=true}
}

\newcommand\whitepage{
\clearpage
\renewcommand\normalcolor{\color{black}}
\pagecolor{white}
\renewcommand\Black{\color{black}}
\color{black}
\hypersetup{colorlinks=true}
}

\whitepage
\setlength{\unitlength}{1mm}

\newenvironment{leftImage}[2][0.3]{%
\begin{minipage}{#1\linewidth}
  \includegraphics[width=\linewidth]{#2}
\end{minipage}\hfil
\begin{minipage}{0.98\linewidth - #1\linewidth}}%
{\end{minipage}}

\newenvironment{rightImage}[2][0.3]{%
\newcommand{\foot}{\begin{minipage}{#1\linewidth}\includegraphics[width=\linewidth]{#2}\end{minipage}}
\begin{minipage}{0.98\linewidth - #1\linewidth}}{\end{minipage}\foot\hfil}


%%% Local Variables: 
%%% mode: latex
%%% TeX-master: t
%%% End: 
