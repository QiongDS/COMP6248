
\documentclass[a4paper]{article}
\usepackage[a4paper,top=2cm,bottom=2cm,left=2cm,right=2cm,marginparwidth=2cm]{geometry}
\usepackage{lmodern}
\usepackage{listings}
\usepackage{amsmath}
\usepackage{amssymb}
\usepackage{bm}
\usepackage{textpos} % package for the positioning
\usepackage{tcolorbox}
\usepackage{pgf, tikz}
\usepackage{url}
\usetikzlibrary{arrows, automata}

\setlength{\parindent}{0em}
\setlength{\parskip}{0.3em}

\usepackage{textcomp}
\begin{document}

\lstset{language=Python,upquote=true}

\setlength{\leftskip}{20pt}
\title{Lab 6 Exercise - Reflections on transfer learning}
\author{Jonathon Hare (jsh2@ecs.soton.ac.uk)}

\maketitle

% \begin{abstract}
% \end{abstract}
% \tableofcontents

This is the exercise that you need to work through \textbf{on your own} after completing the sixth lab session. You'll need to write up your results/answers/findings and submit this to ECS handin as a PDF document along with the other lab exercises near the end of the module (1 pdf document per lab). 

We expect that you \textbf{will use no more than one side} of A4 to cover your responses to \emph{this} exercise; \textbf{answers need only be brief and ideally you will use considerably less space}. This exercise is worth 5\% of your overall module grade.

\section{Transfer Learning}
In the lab notebook you built two different approaches to solving the boat data classification problem utilising a pretrained ResNet50 network. In the first you modified the classification head of the network and performed fine-tuning, whilst in the second you used features from the network to train SVM classifiers.
\\[1em]
\begin{tcolorbox}[title=Finetuning (2 marks)]
Describe your methodology for finetuning your ResNet50 model and provide justification for your choices.
\end{tcolorbox}
\vspace{1em}
\begin{tcolorbox}[title=Reflect on the two different approaches (3 marks)]
Briefly compare the performance of the two approaches. Which one did you find worked better? Which one was fastest? 
\end{tcolorbox}

\end{document}
